Many documents (e.g., academic papers, government reports) are typically
written by multiple authors. While existing tools facilitate and support
such collaborative efforts (e.g., Dropbox, Google Docs), these tools
lack intelligent information sharing mechanisms. Capabilities such as
``track changes'' and ``diff'' visualize changes to authors, but do not
distinguish between minor and major edits and do not consider the
possible effects of edits on other parts of the document. Drawing
collaborators' attention to specific edits and describing them remains
the responsibility of authors. This paper presents our initial work
toward the development of a collaborative system that supports
multi-author writing. We describe methods for tracking paragraphs,
identifying significant edits, and predicting parts of the paper that
are likely to require changes as a result of previous edits. Preliminary
evaluation of the these methods shows promising results.

\section{Introduction}\label{introduction}

Collaborative writing is a common activity for many people: scientists
write papers and proposals together, lawyers draft contracts,
legislators draft legislation. In recent years, collaborative writing of
documents has become even more widespread with the support of frameworks
that simplify distributed editing (e.g., Google Docs, Dropbox). While
document processors provide capabilities such as ``track changes'',
``diff'', and commenting, they lack intelligent information sharing
mechanisms. They do not attempt to reason about the importance of
changes to different authors and do not consider how edits might affect
other parts of the document. As a result, significant coordination
overhead remains: authors need to re-read the entire document frequently
or rely on communication from their co-authors in order to keep track of
the current state of the document and ensure that their edits are
consistent with other edits.

This paper proposes methods for tracking paragraphs across revisions,
identifying significant changes, and predicting paragraphs that are
likely to be edited following a significant edit of a particular
paragraph. The development of these methods is a first step toward the
design of a collaborative system capable of drawing authors' attention
to edits that are important and relevant for them, and of pointing out
other parts of the document that are likely to require further editing.
Systems with such capabilities have the potential to improve coherence
of documents and coordination among authors while reducing the amount of
communication required between authors. An empirical evaluation of the
proposed approach on a corpus of Wikipedia articles shows promising
initial results.

\subsection{Related Work}\label{related-work}

Prior work has studied coordination in collaborative
writing~\cite[interalia]{neuwirth2001computer,kittur2007he}), as well as
the social aspects that arise as a result of edits and comments made by
collaborators~\cite{birnholtz2013write}, and has developed tools for
supporting such collaboration (e.g., Quilt~\cite{fish1988quilt}). Most
closely related to our work are methods and tools for supporting
improved \emph{awareness} of collaborators about document changes. These
include flexible diffing for reporting significant
changes~\cite{neuwirth1992flexible}, methods for categorizing edits and
presenting them to
authors~\cite{fong2010did,papadopoulou2007structured,tam2006framework},
methods for detecting and alerting authors about merge
conflicts~\cite{hainsworth2006enabling}, and methods for detecting
structure in text and help co-authors create coherent documents~
\cite{de2007narrative}. Our approach differs from these prior works in
that it leverages natural language processing (NLP) and data mining
methods to automatically detect significant changes, without requiring
authors to specify explicitly what changes are of interest to them.
Furthermore, these capabilities go beyond prior approaches for
supporting change awareness in that they can predict parts of the
document that have not changed but that are likely to require edits as a
result of other changes.

\section{Approach}\label{approach}

This section describes methods for: (1) tracking paragraphs; (2)
identifying significant changes to paragraphs, and (3) predicting future
changes to the document. Capability (1) is required for both monitoring
changes and predicting them. Capability (2) can help identify important
changes and decide whether to alert authors of a change. Capability (3)
is required in order to draw authors' attention to other parts of the
document that might require changes as a result of a recent edit. We
describe our use of lightweight NLP techniques, which provide the
foundation for these important capabilities.

\subsection{Tracking paragraphs}\label{tracking-paragraphs}

As a document is being edited, paragraphs can be added, moved or
deleted. To track paragraphs throughout document revisions, we developed
a mapping algorithm that determines which paragraph in a new version
corresponds to each of the paragraphs in the previous version (prior to
editing). The algorithm compares paragraphs based on their Levenshtein
distance, that is, the number of changed characters required to move
from one paragraph to another, including additions, deletions, and
substitutions. We use the Levenshtein edit ratio measure, which is
defined as follows:
\(Levenshtein Ratio(a,b) = 1-\frac{Levenshtein Distance(a,b)}{max(|a|,|b|)} \)

A Levenshtein ratio close to 1 indicates high similarity, while a ratio
close to 0 indicates the opposite. For each revision of a document, the
algorithm computes the Levenshtein ratio between each paragraph in the
old version and each paragraph in the new version. Two paragraphs are
mapped across the revision if they each have the highest Levenshtein
ratio with the other. If a paragraph in the new version has no matching
paragraphs in the old version (i.e., none of its ratios were above a
threshold of 0.4), we label the paragraph as an addition. If a paragraph
in the old version is not matched with any paragraphs in the next
version, then we consider it a deletion. Figure~{[}fig:mapping{]}
illustrates this algorithm: some paragraphs found clear matches (e.g.,
0, 1, 2), while paragraph 11 from the old version was deleted and 6, 9,
10, and 12 were added in the new version.

tracking\textsubscript{v}is\textsubscript{p}aper.png

\subsection{Identifying Significant
Changes}\label{identifying-significant-changes}

We consider a significant change to be a noticeable change in the
paragraph's topic and content. To detect such changes, we use a topic
modeling approach which computes the cosine similarity between word
vectors that represent the two versions of a paragraph. Specifically, we
use Latent Semantic Indexing~\cite{deerwester1990indexing} (LSI), which
accounts for synonyms and does not penalize typographical errors. If the
cosine similarity between new and old versions of a paragraph is below a
threshold of 0.8, we consider the edit significant. The threshold of 0.8
was determined empirically by manually evaluating differences between
paragraphs with varied similarity values.

We chose a topic modeling approach for this task because, in contrast
with the Levenshtein ratio approach used for mapping paragraphs, topic
modeling considers the content of the text. To illustrate, if two of a
paragraph's sentences switch places, the Levenshtein ratio would
decrease despite no meaningful change in content. On the other hand,
changing a couple of key words will slightly lower the Levenshtein
ratio, but could drastically affect the content.

\subsection{Predicting Future Edits}\label{predicting-future-edits}

We hypothesized that a paragraph that underwent a significant change
would prompt edits in related paragraphs. Therefore, we investigated
which paragraphs are likely to change in future revisions as a result of
edits to other paragraphs.

We considered three possible types of inter-paragraph relationships: (1)
\emph{proximity}, i.e., neighboring paragraphs; (2) \emph{edit
histories}, i.e., paragraphs that tended to be edited together in
previous revisions, and (3) \emph{topic similarity}, i.e., paragraphs
with similar content. For (2), we labeled pairs of paragraphs that
changed or remained unchanged together in at least 5 of the previous 10
revisions. We chose a window of 10 revisions as it allowed getting a
meaningful signal of correlation between edits, while not looking too
far in the past when the content might have been significantly
different. For (3), we labeled pairs as related if their cosine
similarity was above an empirically determined threshold of 0.4. (There
are rarely paragraphs with a higher similarity than 0.4, while a lower
similarity does not really capture a similarity in content.)

\section{Empirical Evaluation}\label{empirical-evaluation}

This section describes a preliminary evaluation of the proposed methods
for tracking paragraphs, detecting significant edits, and predicting
future changes using a corpus of Wikipedia articles and their revision
histories.

\subsection{Data}\label{data}

We used the complete revision histories of 41 different articles chosen
from a diverse set of topics, ranging from famous people and places to
mathematical algorithms to novels. We removed Wikipedia-specific tags
that indicate formatting and other irrelevant data, and eliminated
versions of articles under 150 characters, as they did not contain
enough text. To focus on revisions that contained a substantial change,
the versions with simple typo fixes (Levenshtein distance \(< 15\) )
were eliminated.

\subsection{Findings}\label{findings}

This section describes findings from a preliminary empirical evaluation.
We focused mostly on the prediction of future edits, but also manually
evaluated paragraph mapping and significant change detection, which form
the basis for edit prediction.

\subsubsection{Detecting Significant
Changes}\label{detecting-significant-changes}

Figure~{[}fig:lsi{]} shows the distribution of LSI similarity between
paragraphs in consecutive versions. As shown, most edits are minor. With
the threshold ratio of 0.8 we labeled about 15\% of the edits as
significant.

\includegraphics{figs/lsiRatios1.pdf}\\

We manually evaluated a random sample of more than 100 random edited
mapped paragraphs to ensure that paragraphs were correctly mapped and
whether the method successfully classified edits as significant or
insignificant. Figure~{[}fig:pars{]} shows an example of a significant
edit: while the bottom paragraph is clearly a revision of the same
paragraph (recurring text shown in bold), the edit is significant
because it adds new content and alters the tone of the text. Overall, we
found that paragraphs were rarely mapped incorrectly (less than
\(5\%\)). Similarly, we found the classification of significant edits to
be correct in most cases, however this is a more subjective measure
which we plan to evaluate further in future work.

\includegraphics{figs/parTopicChangeBigger1.pdf}\\

\subsubsection{Edit Correlations and Prediction of Future
Edits}\label{edit-correlations-and-prediction-of-future-edits}

We confirmed our hypothesis that a significant edit to a paragraph often
triggers edits in other paragraphs in the next revisions. This is
demonstrated in Figure~{[}fig:edits{]}, where each line represents a
paragraph that is edited significantly at time 0. After a period of
relative inactivity, a significant edit triggers further adjustments in
the near future (as represented by downward spikes). We evaluated the
three inter-paragraph relationships (proximity, edit history, and topic
similarity) to determine which paragraphs are more likely to require
further editing after a significant change occurs in the article.

\includegraphics{figs/parChangesNew.pdf}\\

We looked at two measures that might indicate that a paragraph will need
attention. The first checks whether the related paragraphs underwent at
least one significant change within the next ten versions. The second
measure tests whether the two paragraphs are related by edit patterns
ten versions after the significant change, that is, whether they keep
changing together (or remain unchanged together) more than half the
time. This second measure is a stronger indication of a possible
interdependency between the paragraphs than the first measure.

The frequency of occurrences of each type of inter-paragraph
relationship varied. On average, 1.6 pairs of paragraphs related by edit
history were found per revision of the text. A pair related by topic
similarity was only found once every ten versions. Proximity
relationships always exist in each version, as each paragraph has at
least one neighboring paragraph (and most have two).

With respect to the first measure, the relationship of edit history is
much more predictive of a future significant edit than proximity, as
shown in Table~{[}table:futureEdits{]}. We also computed this measure
for the original paragraph as well as randomly sampled paragraphs and
found a significantly lower likelihood than for paragraphs related by
edit similarity. For this measure, the relationship of topic similarity
still needs to be evaluated.

{\textbar{}p{5cm}\textbar{}p{2.5cm}\textbar{}} \textbf{Relationship} \&
\textbf{Percentage}\\Edit History \& 81\%\\

Proximity \& 24\%\\

Paragraph with original change \& 40\%\\

Randomly sampled paragraph \& 15\%\\

With respect to the second measure, we found that 71\% of paragraphs
pairs related by edit history and 63\% for pairs related by topic
similarity continued being modified (or unmodified) together in the next
10 versions. (Proximity was not evaluated for this measure, as it did
almost no better than the random control for the first, less strict,
measure.) Further, we obtained a recall of 80\% with predictions based
on edit history and topic similarity. That is, only 20\% of the
paragraphs that should have been labeled as related (based on
information ten versions in the future) were not included in the set of
related paragraphs predicted by edit history and topic similarity.

\section{Discussion and Future Work}\label{discussion-and-future-work}

Our long-term goal is to build a system that will improve the
collaborative writing process. The methods we developed for tracking
paragraphs throughout document revisions, detecting significant edits,
and predicting future edits provide the basis for such a system. These
capabilities can inform decisions about alerting authors to specific
changes and places in the document that are likely to require revisions.
While the motivation for our work is supporting collaborating authors,
the methods could also provide better support a single author.

In future work, we plan to incorporate information about author identity
to design personalized alerts for authors depending on the edits they
have made. For example, the paragraph tracking algorithm makes it
possible to create a list of authors who have contributed to a specific
paragraph, which can be used to determine who to alert when significant
edits that affect it occur. We also plan to develop additional
algorithms for summarizing changes and investigate the use of algorithms
that measure text coherence (e.g.,Textiling \cite{hearst1994multi}) in
order to alert authors to parts of the text that could be improved.
Finally, we plan to explore alternative interface designs for presenting
the information chosen by the algorithms for authors to consider. We
also plan to incorporate explanations for system recommendations (e.g.,
``we recommend reading the introduction because it is often edited
following edits to the results section'').

\textbf{Acknowledgments.} The work was funded in part by the Nuance
Foundation. We thank Stuart Shieber for helpful discussions.
